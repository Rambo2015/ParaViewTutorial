\chapter{Further Reading}
\label{chap:FurtherReading}

Thank you for participating in this tutorial.  Hopefully you have learned
enough to get you started visualizing large data with ParaView.  Here are
some sources for further reading.

The documentation page on ParaView's web site contains a list of resources
available for further learning and questions.

\begin{reflist}
\item \href{http://www.paraview.org/documentation}{http://www.paraview.org/documentation}
\end{reflist}

The ParaView Guide is a good resource to have with ParaView.  It provides
many other instructions and more detailed descriptions on many features.
% This should be replaced with a direct link when available.
The ParaView guide can be accessed from the ParaView documentation page.

%% \begin{reflist}
%% %% \item  Amy Henderson Squillacote.  \emph{The ParaView Guide}.  Kitware, Inc.,
%% %%   2008. ISBN-10 1-930934-21-1.
%% \item \href{http://paraview.org/Wiki/ParaView/Users_Guide/Table_Of_Contents}{http://paraview.org/Wiki/ParaView/Users\_Guide/Table\_Of\_Contents}
%% \end{reflist}

The ParaView Wiki is full of information that you can use to help you set
up and use ParaView.
\begin{reflist}
\item \href{http://www.paraview.org/Wiki/ParaView}{http://www.paraview.org/Wiki/ParaView}
\end{reflist}
In particular, those of you who wish to install a parallel ParaView server
should consult the appropriate build and install pages.
\begin{reflist}
\item \href{http://www.paraview.org/Wiki/Setting_up_a_ParaView_Server}{http://www.paraview.org/Wiki/Setting\_up\_a\_ParaView\_Server}
\end{reflist}

If you are interested in learning more about visualization or more
specifics about the filters available in ParaView, consider picking up the
following visualization textbook.

\begin{reflist}
\item Will Schroeder, Ken Martin, and Bill Lorensen.  \emph{The
  Visualization Toolkit}.  Kitware, Inc., fourth edition, 2006.  ISBN
  1-930934-19-X.
\end{reflist}

If you plan on customizing ParaView, the previous books and web pages have
lots of information.  For more information about using VTK, the underlying
visualization library, and Qt, the GUI library, consider the following
books have more information.

\begin{reflist}
\item Kitware Inc.  \emph{The VTK User's Guide}.  Kitware, Inc., 2006.
\item Jasmin Blanchette and Mark Summerfield.  \emph{C++ GUI Programming
  with Qt 4}.  Prentice Hall, 2006.  ISBN 0-13-187249-4.
\end{reflist}

If you are interested about the design of parallel visualization and other
features of the VTK pipeline, there are several technical papers available.

\begin{reflist}
\item Kenneth Moreland. ``A Survey of Visualization Pipelines.'' \emph{IEEE
  Transactions on Visualization and Computer Graphics}, 19(3), March 2013.
  DOI~10.1109/TVCG.2012.133.
\item James Ahrens, Charles Law, Will Schroeder, Ken Martin, and Michael
  Papka.  ``A Parallel Approach for Efficiently Visualizing Extremely Large,
  Time-Varying Datasets.''  Technical Report \#LAUR-00-1620, Los Alamos
  National Laboratory, 2000.
\item James Ahrens, Kristi Brislawn, Ken Martin, Berk Geveci, C. Charles
  Law, and Michael Papka.  ``Large-Scale Data Visualization Using Parallel
  Data Streaming.''  \emph{IEEE Computer Graphics and Applications}, 21(4):
  34--41, July/August 2001.
\item Andy Cedilnik, Berk Geveci, Kenneth Moreland, James Ahrens, and Jean
  Farve.  ``Remote Large Data Visualization in the ParaView Framework.''
  \emph{Eurographics Parallel Graphics and Visualization 2006},
  pg. 163--170, May 2006.
\item James P. Ahrens, Nehal Desai, Patrick S. McCormic, Ken Martin, and
  Jonathan Woodring.  ``A Modular, Extensible Visualization System
  Architecture for Culled, Prioritized Data Streaming.''
  \emph{Visualization and Data Analysis 2007, Proceedings of SPIE-IS\&T
    Electronic Imaging}, pg 64950I1-1--12, January 2007.
\item John Biddiscombe, Berk Geveci, Ken Martin, Kenneth Moreland, and
  David Thompson.  ``Time Dependent Processing in a Parallel Pipeline
  Architecture.'' \emph{IEEE Visualization 2007}.  October 2007.
\end{reflist}

If you are interested in the algorithms and architecture for ParaView's
parallel rendering, there are also many technical articles on this as well.

\begin{reflist}
\item Kenneth Moreland, Brian Wylie, and Constantine Pavlakos.  ``Sort-Last
  Parallel Rendering for Viewing Extremely Large Data Sets on Tile
  Displays.''  \emph{Proceedings of IEEE 2001 Symposium on Parallel and
    Large-Data Visualization and Graphics}, pg. 85–92, October 2001.
\item Kenneth Moreland and David Thompson.  ``From Cluster to Wall with
  VTK.''  \emph{Proceddings of IEEE 2003 Symposium on Parallel and
    Large-Data Visualization and Graphics}, pg. 25–31, October 2003.
\item Kenneth Moreland, Lisa Avila, and Lee Ann Fisk.  ``Parallel
  Unstructured Volume Rendering in ParaView.''  \emph{Visualization and Data
  Analysis 2007, Proceedings of SPIE-IS\&T Electronic Imaging},
  pg. 64950F-1–12, January 2007.
\end{reflist}


% Chapter Further Reading
